% \iffalse meta-comment
% !TeX program = xelatex
%
% Copyright (C) 2021 by jiaopjie
%
% This file may be distributed and/or modified under the
% conditions of the LaTeX Project Public License, either
% version 1.3 of this license or (at your option) any later
% version. The latest version of this license is in:
%
% http://www.latex-project.org/lppl.txt
%
% and version 1.3 or later is part of all distributions of
% LaTeX version 2005/12/01 or later.
%
% \fi
%
% \iffalse
%<*internal>
\iffalse
%</internal>
%<*readme>
# 中国计量大学试卷 LaTeX 模板

* 本项目参照 Word 模板设计

* 请及时更新宏包,尤其是 `ctex` 宏包,如果是很早期的版本可能无法运行

* 请使用 `\section` 输入题目标题,得分栏根据 `section` 计数器自动调整

* 题目数最好不要超过 10

* 正确得到密封线、得分栏列数、总页数等需要编译 2 遍

* 本模板作者**不对**使用者的格式问题负责
%</readme>
%<*internal>
\fi
\begingroup
\def\nameoflatex{LaTeX2e}
\expandafter\endgroup\ifx\nameoflatex\fmtname\else
\csname fi\endcsname
%</internal>
%
%<*install>
\input docstrip.tex
\keepsilent

\preamble

Copyright (C) 2021-\the\year by jiaopjie

This file may be distributed and/or modified under the
conditions of the LaTeX Project Public License, either
version 1.3 of this license or (at your option) any later
version. The latest version of this license is in:

http://www.latex-project.org/lppl.txt

and version 1.3 or later is part of all distributions of
LaTeX version 2005/12/01 or later.

\endpreamble

\askforoverwritefalse
\generate{
  \file{\jobname.cls}{\from{\jobname.dtx}{class}}
%</install>
%<*internal>
  \file{\jobname.ins}{\from{\jobname.dtx}{install}}
%</internal>
%<*install>
  \nopreamble\nopostamble
  \file{README.md}{\from{\jobname.dtx}{readme}}
}
\endbatchfile
%</install>
%
%<*internal>
\fi
%</internal>
%
%<*driver>
\ProvidesFile{\jobname.dtx}
%</driver>
%
%<class>\NeedsTeXFormat{LaTeX2e}
%<class>\ProvidesClass{cjluexam}
%<*class>
  [2021/06/19 v1.0.1 CJLU Exam Template]
%</class>
%
%<*driver>
\documentclass[a4paper]{ltxdoc}
\usepackage{ctex,etoolbox,hypdoc,booktabs}
\usepackage{enumitem}
\setlist{nosep}
\AtBeginDocument{\CodelineIndex\EnableCrossrefs}
\AtEndDocument{\PrintIndex}
\AtBeginDocument{\RecordChanges}
\AtEndDocument{\PrintChanges}
\AtBeginEnvironment{verbatim}{\linespread{1}}
\AtBeginEnvironment{macrocode}{\linespread{1}}
\renewcommand\glossaryname{版本历史}
\GlossaryPrologue{\section*{\glossaryname}}
\IndexPrologue{%
  \section*{\indexname}
  \textit{斜体数字表示描述对应索引项的页码;
    带下划线的数字表示定义对应索引项的代码行号;
    罗马字体的数字表示使用对应索引项的代码行号。}}
\setcounter{IndexColumns}{2}
\def\DescribeOpt{\DescribeMacro}
\def\cls{\textsf}
\def\pkg{\textsf}
\def\env{\texttt}
\def\opt{\texttt}
\def\file{\texttt}
\begin{document}
\DocInput{\jobname.dtx}
\linespread{1}
\end{document}
%</driver>
% \fi
%
% \CheckSum{0}
%
% \CharacterTable
% {Upper-case    \A\B\C\D\E\F\G\H\I\J\K\L\M\N\O\P\Q\R\S\T\U\V\W\X\Y\Z
%  Lower-case    \a\b\c\d\e\f\g\h\i\j\k\l\m\n\o\p\q\r\s\t\u\v\w\x\y\z
%  Digits        \0\1\2\3\4\5\6\7\8\9
%  Exclamation   \!      Double quote \"      Hash (number) \#
%  Dollar        \$      Percent      \%      Ampersand     \&
%  Acute accent  \'      Left paren   \(      Right paren   \)
%  Asterisk      \*      Plus         \+      Comma         \,
%  Minus         \-      Point        \.      Solidus       \/
%  Colon         \:      Semicolon    \;      Less than     \<
%  Equals        \=      Greater than \>      Question mark \?
%  Commercial at \@      Left bracket \[      Backslash     \\
%  Right bracket \]      Circumflex   \^      Underscore    \_
%  Grave accent  \`      Left brace   \{      Vertical bar  \|
%  Right brace   \}      Tilde        \~}
%
%
% \changes{v1.0.0}{2021/06/11}{Initial version}
%
% \GetFileInfo{\jobname.dtx}
%
% \DoNotIndex{\mbox,\makebox,\fbox,\framebox,\fboxsep,\parbox}
% \DoNotIndex{\cdot,\rule,\underline,\hline}
% \DoNotIndex{\setmainfont,\setsansfont,\setmonofont}
% \DoNotIndex{\setCJKmainfont,\setCJKsansfont,\setCJKmonofont,\CJKfontspec}
% \DoNotIndex{\fangsong}
% \DoNotIndex{\fontsize,\selectfont,\normalsize,\linespread}
% \DoNotIndex{\normalfont,\bfseries,\itshape,\sffamily,\rmfamily}
% \DoNotIndex{\textbf,\textit,\textsf,\textrm}
% \DoNotIndex{\space,\enskip,\quad,\qquad,\enspace}
% \DoNotIndex{\newpage,\pagebreak,\clearpage,\cleardoublepage}
% \DoNotIndex{\newline,\linebreak,\\,\par}
% \DoNotIndex{\hspace,\hskip,\hfill,\hfil,\fill,\dotfill,\hrulefill}
% \DoNotIndex{\vspace,\vskip,\vfill,\addvspace,\smallskip,\medskip,\bigskip}
% \DoNotIndex{\topskip,\parskip,\baselineskip,\lineskip,\lineskiplimit}
% \DoNotIndex{\noindent,\parindent,\null}
% \DoNotIndex{\setlength,\setcounter}
% \DoNotIndex{\thepage,\arabic}
% \DoNotIndex{\centering,\raggedleft,\raggedright}
% \DoNotIndex{\newcommand,\renewcommand,\providecommand}
% \DoNotIndex{\def,\edef,\let,\protect,\expandafter}
% \DoNotIndex{\newenvironment,\renewenvironment}
% \DoNotIndex{\newif,\ifx,\ifdim,\ifnum,\ifcase,\else,\or,\fi,\relax,\@empty}
% \DoNotIndex{\begin,\end,\bgroup,\egroup,\csname,\endcsname}
% \DoNotIndex{\RequirePackage,\PassOptionsToPackage}
% \DoNotIndex{\DeclareOption,\PassOptionsToClass,\ProcessOptions,\LoadClass}
% \DoNotIndex{\AtBeginDocument,\AtEndDocument}
%
% \title{文档类 \cls{\jobname} 使用说明}
% \author{jiaopjie\thanks{jiaopjie@cjlu.edu.cn}}
% \date{\filedate\qquad\fileversion}
% \maketitle
%
% \section{简介}
%
% \cls{\jobname} 是中国计量大学试卷的 \LaTeX{} 模板.
% 本模板按照《中国计量大学试卷格式》
% \footnote{\url{http://jwc.cjlu.edu.cn/info/1179/4240.htm}}
% 和《中国计量大学试卷参考答案及评分标准》
% \footnote{\url{http://jwc.cjlu.edu.cn/info/1179/4241.htm}}
% 及给出的 Word 模板编写.
% 模板下载地址如下.
% \begin{itemize}
%   \item \url{https://github.com/jiaopjie/cjluexam}
%   \item \url{https://gitee.com/jiaopjie/cjluexam}
% \end{itemize}
%
% \section{编译运行}
%
% \subsection{模板结构}
%
% 本模板包含以下文件.
%
% \begin{center}
%   \begin{tabular}{lll}
%     \toprule
%     类别     & 文件           & 备注\\
%     \midrule
%     模板文件 & \file{\jobname.dtx} & 模板代码文件 (普通用户无需使用)\\
%              & \file{\jobname.cls} & 文档类文件\\
%              & \file{\jobname.ins} & 模板安装文件\\
%              & \file{\jobname.pdf} & 模板说明\\
%              & \file{README.md}    & 简要文本说明\\
%     \midrule
%     示例文档 & \file{exam.tex}     & 试卷源文件\\
%              & \file{exam.pdf}     & 试卷\\
%              & \file{answer.tex}   & 参考答案及评分标准源文件\\
%              & \file{answer.pdf}   & 参考答案及评分标准\\
%     \bottomrule
%   \end{tabular}
% \end{center}
%
% 其中, 文件 \file{\jobname.dtx} 包含的是模板的原始代码.
% 该文件经过编译可以生成其他模板文件.
% 源代码文件仅用于模板维护和了解模板编写细节, 普通用户不必理会.
%
% 文件 \file{exam.tex} 与 \file{answer.tex} 分别是试卷、参考答案的简单示例.
% 编写试卷时在示例文档内进行相应的替换即可.
%
% \subsection{宏包要求}
%
% 本模板直接依赖的宏包有:
% \pkg{ctex},
% \pkg{geometry},
% \pkg{titleps},
% \pkg{lastpage},
% \pkg{tikz},
% \pkg{graphicx},
% \pkg{totcount},
% \pkg{tabularx}.
%
% 编译运行前请尽量将宏包更新到最新版本, 其中 \pkg{ctex} 宏包应升级到 2.0 以上版本.
%
% \subsection{编译}
%
% 推荐使用 XeLaTeX 编译方式.
% 使用 pdfLaTeX/LuaLaTeX 编译方式时, 输出页面会有一些变化.
%
% 得到装订线、座位号的正确位置需要多次编译.
%
% \section{说明}
%
% \subsection{文档选项}
%
% 本模板基于标准文档类 \cls{article} 编写.
%
% \DescribeOpt{answer}
% 模板默认排版试卷,
% 使用文档选项 \opt{answer} 则排版参考答案及评分标准.
%
% \DescribeOpt{twoside}
% \DescribeOpt{oneside}
% 排版试卷时, 模板根据 \opt{twoside}/\opt{oneside} 选项对装订线的位置做了优化.
% \opt{twoside} 时, 在模 4 余 0、1 的页面设装订线;
% \opt{oneside} 时, 在奇数页设装订线.
%
% \begin{center}
%   \begin{tabular}{ll}
%     \toprule
%     选项        & 说明\\
%     \midrule
%     \opt{answer} & 排版参考答案及评分标准\\
%     \midrule
%     \opt{oneside}   & 奇数页设装订线\\
%     \opt{twoside}   & 模 4 余 0、1 的页面设装订线\\
%     \bottomrule
%   \end{tabular}
% \end{center}
%
% \subsection{字体}
%
% 中文字体由 \pkg{ctex} 宏包自动设置.
% 在 XeLaTeX/LuaLaTeX 编译方式下, 可通过 \pkg{fontspec} 宏包的字体选择机制修改字体.
%
% \subsection{标题}
%
% \DescribeMacro{\AcademicYear}
% \DescribeMacro{\Semester}
% \DescribeMacro{\Course}
% \DescribeMacro{\Type}
% \DescribeMacro{\School}
% \DescribeMacro{\ifopen}
% \DescribeMacro{\ifseat}
% \DescribeMacro{\Thing}
% \DescribeMacro{\ExamYear}
% \DescribeMacro{\ExamMonth}
% \DescribeMacro{\ExamDay}
% \DescribeMacro{\ExamTime}
% \DescribeMacro{\Class}
% \DescribeMacro{\Teacher}
% 排版试卷、参考答案时, 生成标题需要收集以下信息.
% \begin{center}
%   \begin{tabular}{lll}
%     \toprule
%     范围      & 宏                  & 说明\\
%     \midrule
%     共用      & \cmd{\AcademicYear} & 学年\\
%               & \cmd{\Semester}     & 学期\\
%               & \cmd{\Course}       & 科目\\
%               & \cmd{\Type}         & 试卷类型\\
%               & \cmd{\School}       & 开课学院\\
%     \midrule
%     试卷      & \cmd{\ifopen}       & 是否开卷\\
%               & \cmd{\ifseat}       & 是否打印座位号\\
%               & \cmd{\Thing}        & 可带入场的物品\\
%               & \cmd{\ExamYear}     & 考试时间-年\\
%               & \cmd{\ExamMonth}    & 考试时间-月\\
%               & \cmd{\ExamDay}      & 考试时间-日\\
%               & \cmd{\ExamTime}     & 考试时间-时\\
%     \midrule
%     参考答案  & \cmd{\Class}        & 学生班级\\
%               & \cmd{\Teacher}      & 教师\\
%     \bottomrule
%   \end{tabular}
% \end{center}
%
% 另外, 排版试卷时, 得分栏的列数根据 \opt{section} 计数器的值自动确定.
% 若输入大题标题时未使用 \cmd{\section} 命令,
% 应设置 \opt{section} 计数器的值, 以确定得分栏列数.
% 例如, 若得分栏中需列 9 个题目, 则应如下设置.
%\begin{verbatim}
%\setcounter{section}{9}
%\end{verbatim}
%
% \subsection{题目标题}
%
% 模板对 \opt{section} 级标题设置了 \opt{runin=true},
% 使得在排版标题后不换行, 直接排版后续内容.
% 将选项 \opt{runin} 置为 \opt{false} 可取消该设置.
%\begin{verbatim}
%\ctexset{\section/runin=false}
%\end{verbatim}
%
% \subsection{项目编号}
%
% 原生的 \env{itemize} 与 \env{enumerate} 环境的各项目之间的距离偏大.
% 可通过宏包 \pkg{enumitem} 进行优化设置.
% 下面是简要的设置示例.
%\begin{verbatim}
%\setlist{nosep,leftmargin=*}
%\end{verbatim}
%
% 宏包 \pkg{tasks} 用于排版每行多项的选项列表,
% 例如: 排版试卷时选择题的选项; 排版答案时选择题、填空题的答案.
% 下面是简要的设置示例.
%\begin{verbatim}
%\settasks{label=(\Alph*),label-width=1.6em,label-offset=.2em,item-indent=1.8em}
%\end{verbatim}
%
% \section{实现代码}
%
%    \begin{macrocode}
%<*class>
%    \end{macrocode}
%
% \subsection{文档选项}
%
% 基于基础文档类 \cls{article} 设计模板.
%
% \DescribeOpt{answer}
% 声明文档选项 \opt{answer}.
%
% \cmd{\ifanswer} 取 \opt{true} 时排版试卷, 取 \opt{false} 时排版参考答案.
%    \begin{macrocode}
\newif\ifanswer
\DeclareOption{answer}{\answertrue}
%    \end{macrocode}
%
% 处理文档选项. 载入 \cls{article} 文档类, 并把其他文档选项传递给它.
% 命令 \cmd{\ProcessOptions} 结束选项声明.
%
%    \begin{macrocode}
\DeclareOption*{\PassOptionsToClass{\CurrentOption}{article}}
\ProcessOptions
\LoadClass{article}
%    \end{macrocode}
%
% \subsection{中文支持}
%
% 若载入 \pkg{fontspec} 宏包, 则对其应用 \opt{no-math} 选项.
% 这样, 通过该宏包修改主文档字体时不修改数学字体.
%
%    \begin{macrocode}
\PassOptionsToPackage{no-math}{fontspec}
%    \end{macrocode}
%
% 载入 \pkg{ctex} 宏包提供中文支持.
% 选项 \opt{heading} 启用标题格式设置功能.
% 选项 \opt{zihao=5} 设置主文档字体尺寸为五号 (10.5\,bp).
%
%    \begin{macrocode}
\RequirePackage[heading,zihao=5]{ctex}[2014/03/06]
%    \end{macrocode}
%
% \subsection{版面}
%
% 载入 \pkg{geometry} 宏包设置页面尺寸.
%
%    \begin{macrocode}
\RequirePackage[a4paper,margin=30mm,footskip=6mm]{geometry}
%    \end{macrocode}
%
% 载入 \pkg{titleps} 宏包设置页眉页脚.
% 其中 \cmd{\setfoot} 的三个参数分别表示左侧、中间、右侧的内容;
%
% 载入 \pkg{lastpage} 宏包打印总页数.
%
% \changes{v1.0.1}{2021/06/19}{改用 \pkg{lastpage} 宏包打印总页数}
%
%    \begin{macrocode}
\RequirePackage{titleps,lastpage}
\ifanswer
  \newpagestyle{main}[\zihao{-5}]{%
    \footrule
    \setfoot{}{\fangsong
      《\Course》课程试卷(\Type)参考答案及评分标准\quad
      第 \thepage 页~共 \pageref{LastPage} 页}{}}
\else
  \newpagestyle{main}[\zihao{-5}]{%
    \footrule
    \setfoot{}{\cjluexam@setbinding\fangsong
      中国计量大学 \AcademicYear 学年 \Semester 学期
      《\Course》课程考试试卷(\Type)
      第 \thepage 页~共 \pageref{LastPage} 页}{}}
\fi
\pagestyle{main}
%    \end{macrocode}
%
% \begin{macro}{\cjluexam@setbinding}
% 设置要设装订线的页面.
%
% 文档选项为 \opt{twoside} 时, 在模 4 余 0、1 的页面设装订线;
% 为 \opt{oneside} 时, 在奇数页设装订线.
%
%    \begin{macrocode}
\if@twoside
  \newcommand\cjluexam@setbinding{%
    \ifcase\numexpr\value{page}-((\value{page}-2)/4)*4\relax
      \cjluexam@binding[1]\or\cjluexam@binding\relax
    \fi
  }
\else
  \newcommand\cjluexam@setbinding
  {\ifodd\value{page}\relax\cjluexam@binding\fi}
\fi
%    \end{macrocode}
% \end{macro}
%
% \begin{macro}{\cjluexam@binding}
% 绘制装订线.
%
% 该命令的参数取 \opt{-1} 时装订线在左侧, 参数取 \opt{1} 时在右侧.
% 得到装订线的正确位置需要多次编译.
%
%    \begin{macrocode}
\RequirePackage{graphicx,tikz}
\newcommand\cjluexam@binding[1][-1]{%
  \begin{tikzpicture}[remember picture,overlay]
    \zihao{5}
    \node
      [
        text width=\textheight,
        rotate=-90,
        yshift=.5*#1*\textwidth+#1*.5cm
      ]
      at (current page.center)
      {
        \sffamily\dotfill
        \raisebox{-.45em}{ \rotatebox{90}{装} }\dotfill
        \raisebox{-.45em}{ \rotatebox{90}{订} }\dotfill
        \raisebox{-.45em}{ \rotatebox{90}{线} }\dotfill
      };
  \end{tikzpicture}%
}
%    \end{macrocode}
% \end{macro}
%
% \subsection{标题}
%
% \begin{macro}{\AcademicYear}
% \begin{macro}{\Semester}
% \begin{macro}{\Course}
% \begin{macro}{\Type}
% \begin{macro}{\School}
% 排版试卷、参考答案时, 生成标题都需要收集的信息.
%
%    \begin{macrocode}
\def\AcademicYear{} % 学年
\def\Semester{}     % 学期
\def\Course{}       % 科目
\def\Type{}         % 试卷类型
\def\School{}       % 开课学院
%    \end{macrocode}
% \end{macro}
% \end{macro}
% \end{macro}
% \end{macro}
% \end{macro}
%
% \begin{macro}{\ifopen}
% \begin{macro}{\ifseat}
% \begin{macro}{\Thing}
% \begin{macro}{\ExamYear}
% \begin{macro}{\ExamMonth}
% \begin{macro}{\ExamDay}
% \begin{macro}{\ExamTime}
% 排版试卷时, 生成标题需要收集的信息.
%
%    \begin{macrocode}
\newif\ifopen       % 是否开卷
\newif\ifseat       % 是否打印座位号
\def\Thing{}        % 可带入场的物品
\def\ExamYear{}     % 考试时间-年
\def\ExamMonth{}    % 考试时间-月
\def\ExamDay{}      % 考试时间-日
\def\ExamTime{}     % 考试时间-时
%    \end{macrocode}
% \end{macro}
% \end{macro}
% \end{macro}
% \end{macro}
% \end{macro}
% \end{macro}
% \end{macro}
%
% \begin{macro}{\Class}
% \begin{macro}{\Teacher}
% 排版参考答案时, 生成标题需要收集的信息.
%
%    \begin{macrocode}
\def\Class{}        % 学生班级
\def\Teacher{}      % 教师
%    \end{macrocode}
% \end{macro}
% \end{macro}
%
% \begin{macro}{\maketitle}
% 重定义 \cmd{\maketitle} 命令输出标题、考生相关信息、得分栏、座位号等.
%
%    \begin{macrocode}
\RequirePackage{totcount}
\regtotcounter{section}
\ifanswer
  \renewcommand{\maketitle}{
    \bgroup
    \parindent0pt\linespread{1.5}\zihao{-4}%
    \begin{center}
      \zihao{4}\bfseries\fangsong
      中国计量大学 \AcademicYear 学年第 \Semester 学期\\
      《\Course》课程试卷(\Type)\\
      参考答案及评分标准
    \end{center}%
    开课二级学院:\Fill{9em}{\School},
    学生班级:\Fill{5em}{\Class},
    教师:\Fill{6em}{\Teacher}
    \egroup
  }
\else
  \renewcommand{\maketitle}{%
    \bgroup
    \parindent0pt\linespread{1.5}\zihao{-4}\setlength{\fboxsep}{-0.4pt}%
    \begin{center}
      \ifseat\cjluexam@seat\fi
      \zihao{4}\bfseries\fangsong
      中国计量大学 \AcademicYear 学年第 \Semester 学期\\
      《\Course》课程考试试卷(\Type)
    \end{center}%
    开课二级学院:\Fill{9em}{\School},%
    考试时间:\mbox{%
      \Fill{3em}{\ExamYear}年\Fill{2em}{\ExamMonth}月%
      \Fill{2em}{\ExamDay }日\Fill{2em}{\ExamTime }时}\\
    考试形式:%
    \ifopen
      闭卷 \fbox{\rule[0.683em]{0.683em}{0em}}、%
      开卷 \rule{0.683em}{0.683em},%
    \else
      闭卷 \rule{0.683em}{0.683em}、%
      开卷 \fbox{\rule[0.683em]{0.683em}{0em}},%
    \fi
    \mbox{允许带\Fill{17em}{\Thing}入场}\\
    考生姓名:\Fill{5.2em}{}
    学号:\Fill{5.2em}{}
    专业:\Fill{5.2em}{}
    班级:\Fill{5.2em}{}
    \par\smallskip
    \edef\cjluexam@num{\totvalue{section}}%
    \ifnum\cjluexam@num<1\def\cjluexam@num{1}\fi
    \cjluexam@score{\cjluexam@num}%
    \egroup
  }
\fi
%    \end{macrocode}
% \end{macro}
%
% \begin{macro}{\Fill}
% \begin{macro}{\cjluexam@length}
% 输出有指定长度的下划线的文本.
%
% 命令 \cmd{\Fill} 的参数 \opt{\#1} 是下划线的指定长度,
% 参数 \opt{\#2} 是要输出的文本.
% 实现效果: 文本长度小于指定长度时, 文本居中; 否则, 文本左对齐.
%
% 辅助长度变量 \cmd{\cjluexam@length} 用于条件判断.
%
%    \begin{macrocode}
\newlength\cjluexam@length
\newcommand\Fill[2]{%
  \settowidth{\cjluexam@length}{#2}%
  \ifdim\cjluexam@length>#1
    \underline{\mbox{#2}}%
  \else
    \underline{\makebox[#1]{#2}}%
  \fi
}
%    \end{macrocode}
% \end{macro}
% \end{macro}
%
% \begin{macro}{\cjluexam@score}
% 生成列数可变的得分栏.
%
%    \begin{macrocode}
\RequirePackage{tabularx}
\newcommand\cjluexam@score[1]{%
  \ifnum #1<7
    \begin{tabularx}{\textwidth}{|*{\numexpr#1+2}{>{\hfil}X|}}
      \hline
      题序   & \ccjluexam@Ncell{1}{#1}{2} & 总分 \\\hline
      得分   & \ccjluexam@Ncell{1}{#1}{0} &      \\\hline
      评卷人 & \ccjluexam@Ncell{1}{#1}{0} &      \\\hline
    \end{tabularx}%
  \else
    \ifnum #1<11
      \begin{tabularx}{\textwidth}{|c|*{\numexpr#1+1}{@{}>{\hfil}X@{}|}}
        \hline
        题序   & \ccjluexam@Ncell{1}{#1}{2} & 总分 \\\hline
        得分   & \ccjluexam@Ncell{1}{#1}{0} &      \\\hline
        评卷人 & \ccjluexam@Ncell{1}{#1}{0} &      \\\hline
      \end{tabularx}%
    \else
      \begin{tabular}{|c|*{#1}{@{}>{\hfil}p{2.55em}@{}|}c|}
        \hline
        题序   & \ccjluexam@Ncell{1}{#1}{2} & 总分 \\\hline
        得分   & \ccjluexam@Ncell{1}{#1}{0} &      \\\hline
        评卷人 & \ccjluexam@Ncell{1}{#1}{0} &      \\\hline
      \end{tabular}%
    \fi
  \fi
}
%    \end{macrocode}
% \end{macro}
%
% \begin{macro}{\ccjluexam@Ncell}
% \begin{macro}{\cjluexam@gobble}
% 生成指定列数的单元格的格式字符串.
%
% 命令 \cmd{\ccjluexam@Ncell} 的参数 \opt{\#1} 一定要取 \opt{1}, 表示起始序号.
% 参数 \opt{\#2} 是指定的列数;
% 参数 \opt{\#3} 取 \opt{0} 时表示单元格内部无内容;
% 取 \opt{1} 时输出递增数列;
% 取其他值时输出中文数字形式的递增数列.
%
% 辅助命令 \cmd{\cjluexam@gobble} 的功能就是丢弃它的参数.
%
% 该方案参考了
% \href{https://tex.stackexchange.com/questions/165625/how-to-fill-a-dynamically-generated-table-with-dynamic-content}{StackExchange}
% 与
% \href{https://www.overleaf.com/learn/latex/Articles/How_does_%5Cexpandafter_work:_A_detailed_macro_case_study}{Overleaf}
% 的代码.
% 其他方案可参考\href{https://zhuanlan.zhihu.com/p/58103339}{知乎}.
%
%    \begin{macrocode}
\def\cjluexam@gobble#1{}
\newcommand\ccjluexam@Ncell[3]{%
  \ifnum  #1<2 \expandafter\cjluexam@gobble\fi &
  \ifcase #3 \or #1\else\zhnumber{#1}\fi
  \ifnum  #1<#2
    \expandafter\ccjluexam@Ncell
    \expandafter{\the\numexpr1+#1\expandafter}%
    \expandafter{\the\numexpr  #2\expandafter}%
    \expandafter{\the\numexpr  #3\expandafter}%
  \fi
}
%    \end{macrocode}
% \end{macro}
% \end{macro}
%
% \begin{macro}{\cjluexam@seat}
% 打印座位号.
%
% 下面的坐标以页面左上角为原点.
% 得到座位号的正确位置需要多次编译.
%
%    \begin{macrocode}
\RequirePackage{tikz}
\newcommand\cjluexam@seat{%
  \begin{tikzpicture}
    [remember picture,overlay,ampersand replacement=\&]
    \zihao{-4}
    \matrix[nodes={draw,inner ysep=8pt}]
      at ([shift={(25mm,-25mm)}]current page.north west)
      {\node{\phantom{座位号}}; \& \node{座位号};\\};
  \end{tikzpicture}%
}
%    \end{macrocode}
% \end{macro}
%
% \subsection{题目标题}
%
% 设置 \cmd{\section} 级标题的格式.
% 选项 \opt{runin} 取 \opt{true} 时, 标题后不换行, 直接排版后续内容.
%
%    \begin{macrocode}
\ctexset{section={
  name={,、},
  number=\chinese{section},
  format=\sffamily,
  aftername={},
  aftertitle={},
  beforeskip=2ex plus .5ex minus .5ex,
  afterskip=1ex plus .5ex minus .5ex,
  runin=true,
}}
%    \end{macrocode}
%
% \subsection{其他}
%
% 两行之间的最小空白.
%    \begin{macrocode}
\setlength\lineskiplimit{.333em}
\setlength\lineskip{.333em}
%    \end{macrocode}
%
%    \begin{macrocode}
%</class>
%    \end{macrocode}
%
% \Finale
\endinput
