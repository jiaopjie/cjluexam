% \iffalse meta-comment
% !TeX program = xelatex
%
% Copyright (C) 2021 by jiaopjie
%
% This file may be distributed and/or modified under the
% conditions of the LaTeX Project Public License, either
% version 1.3 of this license or (at your option) any later
% version. The latest version of this license is in:
%
% http://www.latex-project.org/lppl.txt
%
% and version 1.3 or later is part of all distributions of
% LaTeX version 2005/12/01 or later.
%
% \fi
%
% \iffalse
%<*internal>
\iffalse
%</internal>
%<*readme>
# 中国计量大学试卷 LaTeX 模板

* 本项目参照 Word 模板设计

* 请使用 XeLaTeX 编译方式, 使用其他编译方式时, 页面会有所变化

* `ctex` 宏包应升级到 2.0 以上版本

* 得分栏列数根据 `section` 计数器自动调整, 请使用 `\section` 命令输入题目标题

* 题目数最好不要超过 10

* 正确得到装订线、座位号、得分栏、总页数等需要编译 2 遍

* 本模板作者**不对**使用者的格式问题负责
%</readme>
%<*internal>
\fi
\begingroup
\def\nameoflatex{LaTeX2e}
\expandafter\endgroup\ifx\nameoflatex\fmtname\else
\csname fi\endcsname
%</internal>
%
%<*install>
\input docstrip.tex
\keepsilent

\preamble

Copyright (C) 2021-\the\year by jiaopjie

This file may be distributed and/or modified under the
conditions of the LaTeX Project Public License, either
version 1.3 of this license or (at your option) any later
version. The latest version of this license is in:

http://www.latex-project.org/lppl.txt

and version 1.3 or later is part of all distributions of
LaTeX version 2005/12/01 or later.

\endpreamble

\askforoverwritefalse
\generate{
  \file{\jobname.cls}{\from{\jobname.dtx}{class}}
%</install>
%<*internal>
  \file{\jobname.ins}{\from{\jobname.dtx}{install}}
%</internal>
%<*install>
  \nopreamble\nopostamble
  \file{README.md}{\from{\jobname.dtx}{readme}}
}
\endbatchfile
%</install>
%
%<*internal>
\fi
%</internal>
%
%<*driver>
\ProvidesFile{\jobname.dtx}
%</driver>
%
%<class>\NeedsTeXFormat{LaTeX2e}
%<class>\ProvidesClass{cjluexam}
%<*class>
  [2021/12/01 v1.0.2 CJLU Exam Template]
%</class>
%
%<*driver>
\documentclass[a4paper]{ltxdoc}
\usepackage{ctex,etoolbox,hypdoc,booktabs}
\usepackage{enumitem}
\setlist{nosep}
\AtBeginDocument{\CodelineIndex\EnableCrossrefs}
\AtEndDocument{\PrintIndex}
\AtBeginDocument{\RecordChanges}
\AtEndDocument{\PrintChanges}
\AtBeginEnvironment{verbatim}{\linespread{1}}
\AtBeginEnvironment{macrocode}{\linespread{1}}
\renewcommand\glossaryname{版本历史}
\GlossaryPrologue{\section*{\glossaryname}}
\IndexPrologue{%
  \section*{\indexname}
  \textit{斜体数字表示描述对应索引项的页码;
    带下划线的数字表示定义对应索引项的代码行号;
    罗马字体的数字表示使用对应索引项的代码行号。}}
\setcounter{IndexColumns}{2}
\def\DescribeOpt{\DescribeMacro}
\def\cls{\textsf}
\def\pkg{\textsf}
\def\env{\texttt}
\def\opt{\texttt}
\def\file{\texttt}
\begin{document}
\DocInput{\jobname.dtx}
\linespread{1}
\end{document}
%</driver>
% \fi
%
% \CheckSum{0}
%
% \CharacterTable
% {Upper-case    \A\B\C\D\E\F\G\H\I\J\K\L\M\N\O\P\Q\R\S\T\U\V\W\X\Y\Z
%  Lower-case    \a\b\c\d\e\f\g\h\i\j\k\l\m\n\o\p\q\r\s\t\u\v\w\x\y\z
%  Digits        \0\1\2\3\4\5\6\7\8\9
%  Exclamation   \!      Double quote \"      Hash (number) \#
%  Dollar        \$      Percent      \%      Ampersand     \&
%  Acute accent  \'      Left paren   \(      Right paren   \)
%  Asterisk      \*      Plus         \+      Comma         \,
%  Minus         \-      Point        \.      Solidus       \/
%  Colon         \:      Semicolon    \;      Less than     \<
%  Equals        \=      Greater than \>      Question mark \?
%  Commercial at \@      Left bracket \[      Backslash     \\
%  Right bracket \]      Circumflex   \^      Underscore    \_
%  Grave accent  \`      Left brace   \{      Vertical bar  \|
%  Right brace   \}      Tilde        \~}
%
%
% \changes{v1.0.0}{2021/06/11}{Initial version}
%
% \GetFileInfo{\jobname.dtx}
%
% \DoNotIndex{\mbox,\makebox,\fbox,\framebox,\fboxsep,\parbox}
% \DoNotIndex{\cdot,\rule,\underline,\hline}
% \DoNotIndex{\setmainfont,\setsansfont,\setmonofont}
% \DoNotIndex{\setCJKmainfont,\setCJKsansfont,\setCJKmonofont,\CJKfontspec}
% \DoNotIndex{\fangsong}
% \DoNotIndex{\fontsize,\selectfont,\normalsize,\linespread}
% \DoNotIndex{\normalfont,\bfseries,\itshape,\sffamily,\rmfamily}
% \DoNotIndex{\textbf,\textit,\textsf,\textrm}
% \DoNotIndex{\space,\enskip,\quad,\qquad,\enspace}
% \DoNotIndex{\newpage,\pagebreak,\clearpage,\cleardoublepage}
% \DoNotIndex{\newline,\linebreak,\\,\par}
% \DoNotIndex{\hspace,\hskip,\hfill,\hfil,\fill,\dotfill,\hrulefill}
% \DoNotIndex{\vspace,\vskip,\vfill,\addvspace,\smallskip,\medskip,\bigskip}
% \DoNotIndex{\topskip,\parskip,\baselineskip,\lineskip,\lineskiplimit}
% \DoNotIndex{\noindent,\parindent,\null}
% \DoNotIndex{\setlength,\setcounter}
% \DoNotIndex{\thepage,\arabic}
% \DoNotIndex{\centering,\raggedleft,\raggedright}
% \DoNotIndex{\newcommand,\renewcommand,\providecommand}
% \DoNotIndex{\def,\edef,\let,\protect,\expandafter}
% \DoNotIndex{\newenvironment,\renewenvironment}
% \DoNotIndex{\newif,\ifx,\ifdim,\ifnum,\ifcase,\else,\or,\fi,\relax,\@empty}
% \DoNotIndex{\begin,\end,\begingroup,\endgroup,\csname,\endcsname}
% \DoNotIndex{\RequirePackage,\PassOptionsToPackage}
% \DoNotIndex{\DeclareOption,\PassOptionsToClass,\ProcessOptions,\LoadClass}
% \DoNotIndex{\AtBeginDocument,\AtEndDocument}
%
% \title{文档类 \cls{\jobname} 使用说明}
% \author{jiaopjie\thanks{jiaopjie@cjlu.edu.cn}}
% \date{\filedate\qquad\fileversion}
% \maketitle
%
% \section{简介}
%
% \cls{\jobname} 是中国计量大学试卷的 \LaTeX{} 模板.
% 本模板按照《中国计量大学试卷格式》
% \footnote{\url{http://jwc.cjlu.edu.cn/info/1179/4240.htm}}
% 和《中国计量大学试卷参考答案及评分标准》
% \footnote{\url{http://jwc.cjlu.edu.cn/info/1179/4241.htm}}
% 及给出的 Word 模板编写.
% 下载地址如下.
% \begin{itemize}
%   \item \url{https://github.com/jiaopjie/cjluexam}
%   \item \url{https://gitee.com/jiaopjie/cjluexam}
% \end{itemize}
%
% \subsection{宏包要求}
%
% 本模板直接依赖的宏包有:
% \pkg{ctex},
% \pkg{geometry},
% \pkg{titleps},
% \pkg{lastpage},
% \pkg{tikz},
% \pkg{graphicx},
% \pkg{totcount},
% \pkg{tabularx}.
%
% \subsection{注意事项}
%
% \begin{itemize}
%   \item
%     请使用 XeLaTeX 编译方式, 使用其他编译方式时, 页面会有所变化.
%   \item
%     \pkg{ctex} 宏包应升级到 2.0 以上版本.
%   \item
%     得分栏列数根据 \opt{section} 计数器自动调整,
%     请使用 \cmd{\section} 命令输入题目标题.
%   \item
%     题目数最好不要超过 10.
%   \item
%     正确得到装订线、座位号、得分栏、总页数等需要编译 2 遍.
% \end{itemize}
%
% \subsection{模板结构}
%
% 本模板由模板文件和示例文档组成, 如表~\ref{tab:files} 所示.
% 编写试卷时在示例文档内进行相应的替换即可.
%
% \begin{table}[htbp]
%   \caption{模板结构}
%   \label{tab:files}
%   \centering
%   \begin{tabular}{lll}
%     \toprule
%     类别     & 文件           & 说明\\
%     \midrule
%     模板文件 & \file{\jobname.dtx} & 模板代码文件\\
%              & \file{\jobname.cls} & 文档类文件\\
%              & \file{\jobname.ins} & 模板安装文件\\
%              & \file{\jobname.pdf} & 模板说明\\
%              & \file{README.md}    & 简要文本说明\\
%     \midrule
%     示例文档 & \file{exam.tex}     & 试卷源文件\\
%              & \file{exam.pdf}     & 试卷\\
%              & \file{answer.tex}   & 参考答案及评分标准源文件\\
%              & \file{answer.pdf}   & 参考答案及评分标准\\
%     \bottomrule
%   \end{tabular}
% \end{table}
%
% \section{说明}
%
% 本模板基于标准文档类 \cls{article} 编写.
%
% \subsection{文档选项}
%
% \DescribeOpt{answer}
% 模板默认排版试卷,
% 使用文档选项 \opt{answer} 则排版参考答案及评分标准.
%
% \DescribeOpt{twoside}
% \DescribeOpt{oneside}
% 排版试卷时, 模板根据 \opt{twoside}/\opt{oneside} 选项对装订线的位置做了优化.
% \opt{twoside} 时, 在模 4 余 0、1 的页面设装订线;
% \opt{oneside} 时, 在奇数页设装订线.
%
% \begin{table}[htbp]
%   \caption{文档选项}
%   \label{tab:opt}
%   \centering
%   \begin{tabular}{ll}
%     \toprule
%     选项            & 说明\\
%     \midrule
%     \opt{answer}    & 排版参考答案及评分标准\\
%     \midrule
%     \opt{oneside}   & 装订线设在奇数页\\
%     \opt{twoside}   & 装订线设在模 4 余 0、1 的页面\\
%     \bottomrule
%   \end{tabular}
% \end{table}
%
% \clearpage
%
% \subsection{试卷信息}
%
% \DescribeMacro{\AcademicYear}
% \DescribeMacro{\Term}
% \DescribeMacro{\Course}
% \DescribeMacro{\Type}
% \DescribeMacro{\College}
% \DescribeMacro{\ifopen}
% \DescribeMacro{\ifseat}
% \DescribeMacro{\Thing}
% \DescribeMacro{\TestYear}
% \DescribeMacro{\TestMonth}
% \DescribeMacro{\TestDay}
% \DescribeMacro{\TestTime}
% \DescribeMacro{\Class}
% \DescribeMacro{\Teacher}
% 排版试卷、参考答案时, 生成标题、页脚需要收集表~\ref{tab:info} 中的信息.
%
% \begin{table}[htbp]
%   \caption{试卷信息}
%   \label{tab:info}
%   \centering
%   \begin{tabular}{lll}
%     \toprule
%     范围      & 宏                  & 说明\\
%     \midrule
%     共用      & \cmd{\AcademicYear} & 学年\\
%               & \cmd{\Term}         & 学期\\
%               & \cmd{\Course}       & 科目\\
%               & \cmd{\Type}         & 试卷类型\\
%               & \cmd{\College}      & 开课学院\\
%     \midrule
%     试卷      & \cmd{\ifopen}       & 是否开卷\\
%               & \cmd{\ifseat}       & 是否打印座位号\\
%               & \cmd{\Thing}        & 可带入场的物品\\
%               & \cmd{\TestYear}     & 考试时间-年\\
%               & \cmd{\TestMonth}    & 考试时间-月\\
%               & \cmd{\TestDay}      & 考试时间-日\\
%               & \cmd{\TestTime}     & 考试时间-时\\
%     \midrule
%     参考答案  & \cmd{\Class}        & 学生班级\\
%               & \cmd{\Teacher}      & 教师\\
%     \bottomrule
%   \end{tabular}
% \end{table}
%
% 得分栏列数根据 \opt{section} 计数器自动调整.
% 若输入题目标题时未使用 \cmd{\section} 命令,
% 应手动设置 \opt{section} 计数器的值.
%\begin{verbatim}
%\setcounter{section}{9}
%\end{verbatim}
%
% \subsection{题目标题}
%
% 模板设置 \opt{section} 为无衬线字体; 标题后不换行, 直接排版后续内容.
% 以下命令分别抵消这两个设置.
%\begin{verbatim}
%\ctexset{section/format={}}
%\ctexset{section/runin=false}
%\end{verbatim}
%
% \subsection{字体}
%
% 中文字体由 \pkg{ctex} 宏包自动设置.
% 在 XeLaTeX/LuaLaTeX 编译方式下, 可通过 \pkg{fontspec} 宏包的字体选择机制修改字体.
%\begin{verbatim}
%\setCJKmainfont{SimSun}
%\setCJKsansfont{SimHei}
%\end{verbatim}
%
% \subsection{项目编号}
%
% 原生的 \env{itemize} 与 \env{enumerate} 环境的各项目之间的距离偏大.
% 可通过宏包 \pkg{enumitem} 进行优化设置.
%\begin{verbatim}
%\setlist{nosep,leftmargin=*}
%\end{verbatim}
%
% 选择题的选项可使用 \pkg{tasks} 宏包提供的 \env{tasks} 环境.
%\begin{verbatim}
%\settasks{label=(\Alph*),label-width=1.6em,label-offset=.2em,item-indent=1.8em}
%\settasks{before-skip=0pt,after-skip=0pt,after-item-skip=0pt}
%\end{verbatim}
%
% \section{实现代码}
%
%    \begin{macrocode}
%<*class>
%    \end{macrocode}
%
% \subsection{文档选项}
%
% 基于基础文档类 \cls{article} 设计模板.
%
% \DescribeOpt{answer}
% 声明文档选项 \opt{answer}.
%
% \cmd{\ifanswer} 取 \opt{true} 时排版参考答案, 取 \opt{false} 时排版试卷.
%    \begin{macrocode}
\newif\ifanswer
\DeclareOption{answer}{\answertrue}
%    \end{macrocode}
%
% 处理文档选项. 载入 \cls{article} 文档类, 并把其他文档选项传递给它.
% 命令 \cmd{\ProcessOptions} 结束选项声明.
%
%    \begin{macrocode}
\DeclareOption*{\PassOptionsToClass{\CurrentOption}{article}}
\ProcessOptions
\LoadClass{article}
%    \end{macrocode}
%
% \subsection{中文支持}
%
% 若载入 \pkg{fontspec} 宏包, 则对其使用 \opt{no-math} 选项.
% 这样, 通过该宏包修改主文档字体时不修改数学字体.
%
%    \begin{macrocode}
\PassOptionsToPackage{no-math}{fontspec}
%    \end{macrocode}
%
% 载入 \pkg{ctex} 宏包提供中文支持.
% 选项 \opt{heading} 启用标题格式设置功能.
% 选项 \opt{zihao=5} 设置主文档字体尺寸为五号 (10.5\,bp).
%
%    \begin{macrocode}
\RequirePackage[heading,zihao=5]{ctex}[2014/03/06]
%    \end{macrocode}
%
% \subsection{版面}
%
% 载入 \pkg{geometry} 宏包设置页面尺寸.
%
%    \begin{macrocode}
\RequirePackage[a4paper,margin=30mm,footskip=6mm]{geometry}
%    \end{macrocode}
%
% \begin{macro}{\cjluexam@foot}
% 页脚内容.
% 其中, 总页数依赖 \pkg{lastpage} 宏包,
% \cmd{\cjluexam@binding} 设装订线.
%
% \changes{v1.0.2}{2021/11/15}{改用宏命令储存页脚内容}
%
%    \begin{macrocode}
\RequirePackage{lastpage}
\ifanswer
  \def\cjluexam@foot
  {《\Course》课程试卷(\Type)参考答案及评分标准\quad
  第 \thepage 页~共 \pageref{LastPage} 页}
\else
  \def\cjluexam@foot
  {\cjluexam@binding 中国计量大学 \AcademicYear 学年第 \Term 学期%
  《\Course》课程考试试卷(\Type)第 \thepage 页~共 \pageref{LastPage} 页}
\fi
%    \end{macrocode}
% \end{macro}
%
% 载入 \pkg{titleps} 宏包设置页眉页脚.
% 其中 \cmd{\setfoot} 的三个参数分别表示左侧、中间、右侧的内容;
%
% \changes{v1.0.1}{2021/06/19}{改用 \pkg{lastpage} 宏包打印总页数}
%
%    \begin{macrocode}
\RequirePackage{titleps}
\newpagestyle{main}[\zihao{-5}]
  {\footrule\setfoot{}{\fangsong\cjluexam@foot}{}}
\pagestyle{main}
%    \end{macrocode}
%
% \begin{macro}{\cjluexam@binding}
% 设装订线.
% 文档选项为 \opt{twoside} 时, 装订线设在模 4 余 0、1 的页面;
% 为 \opt{oneside} 时, 装订线设在奇数页.
%
%    \begin{macrocode}
\if@twoside
  \newcommand\cjluexam@binding{%
    \ifcase\numexpr\value{page}-((\value{page}-2)/4)*4\relax
      \cjluexam@drawbinding[1]\or\cjluexam@drawbinding\relax
    \fi
  }
\else
  \newcommand\cjluexam@binding
  {\ifodd\value{page}\relax\cjluexam@drawbinding\fi}
\fi
%    \end{macrocode}
% \end{macro}
%
% \begin{macro}{\cjluexam@drawbinding}
% 装订线的具体绘制.
% 参数取 \opt{-1} 时装订线在左侧, 参数取 \opt{1} 时在右侧.
%
%    \begin{macrocode}
\RequirePackage{graphicx,tikz}
\newcommand\cjluexam@drawbinding[1][-1]{%
  \begin{tikzpicture}[remember picture,overlay]
    \zihao{5}\sffamily
    \path (current page.center)
      node
      [
        text width=\textheight,
        rotate=-90,
        yshift=.5*#1*\textwidth+#1*.5cm,
      ]
      {%
        \dotfill\raisebox{-.45em}{\rotatebox{90}{装}}%
        \dotfill\raisebox{-.45em}{\rotatebox{90}{订}}%
        \dotfill\raisebox{-.45em}{\rotatebox{90}{线}}\dotfill
      };
  \end{tikzpicture}%
}
%    \end{macrocode}
% \end{macro}
%
% \subsection{标题}
%
% \changes{v1.0.2}{2021/11/29}{更改学期、学院、考试时间的宏名}
% \begin{macro}{\AcademicYear}
% \begin{macro}{\Term}
% \begin{macro}{\Course}
% \begin{macro}{\Type}
% \begin{macro}{\College}
% 排版试卷、参考答案时, 生成标题都需要收集的信息.
%
%    \begin{macrocode}
\def\AcademicYear{} % 学年
\def\Term{}         % 学期
\def\Course{}       % 科目
\def\Type{}         % 试卷类型
\def\College{}      % 开课学院
%    \end{macrocode}
% \end{macro}
% \end{macro}
% \end{macro}
% \end{macro}
% \end{macro}
%
% \begin{macro}{\ifopen}
% \begin{macro}{\ifseat}
% \begin{macro}{\Thing}
% \begin{macro}{\TestYear}
% \begin{macro}{\TestMonth}
% \begin{macro}{\TestDay}
% \begin{macro}{\TestTime}
% 排版试卷时, 生成标题需要收集的信息.
%
%    \begin{macrocode}
\newif\ifopen       % 是否开卷
\newif\ifseat       % 是否打印座位号
\def\Thing{}        % 可带入场的物品
\def\TestYear{}     % 考试时间-年
\def\TestMonth{}    % 考试时间-月
\def\TestDay{}      % 考试时间-日
\def\TestTime{}     % 考试时间-时
%    \end{macrocode}
% \end{macro}
% \end{macro}
% \end{macro}
% \end{macro}
% \end{macro}
% \end{macro}
% \end{macro}
%
% \begin{macro}{\Class}
% \begin{macro}{\Teacher}
% 排版参考答案时, 生成标题需要收集的信息.
%
%    \begin{macrocode}
\def\Class{}        % 学生班级
\def\Teacher{}      % 教师
%    \end{macrocode}
% \end{macro}
% \end{macro}
%
% \begin{macro}{\maketitle}
% 重定义 \cmd{\maketitle} 命令输出标题、考生相关信息、得分栏、座位号等.
%
%    \begin{macrocode}
\RequirePackage{totcount}
\regtotcounter{section}
\ifanswer
  \renewcommand{\maketitle}{%
    \begingroup
    \parindent0pt\linespread{1.5}\zihao{-4}%
    \begin{center}
      \zihao{4}\bfseries\fangsong
      中国计量大学 \AcademicYear 学年第 \Term 学期\\
      《\Course》课程试卷(\Type)\\
      参考答案及评分标准
    \end{center}%
    开课二级学院:\Fill{9em}{\College},
    学生班级:\Fill{5em}{\Class},
    教师:\Fill{6em}{\Teacher}
    \endgroup
  }
\else
  \renewcommand{\maketitle}{%
    \begingroup
    \parindent0pt\linespread{1.5}\zihao{-4}%
    \begin{center}
      \ifseat\cjluexam@seat\fi
      \zihao{4}\bfseries\fangsong
      中国计量大学 \AcademicYear 学年第 \Term 学期\\
      《\Course》课程考试试卷(\Type)
    \end{center}%
    开课二级学院:\Fill{9em}{\College},%
    考试时间:\mbox{%
      \Fill{3em}{\TestYear}年\Fill{2em}{\TestMonth}月%
      \Fill{2em}{\TestDay }日\Fill{2em}{\TestTime }时}\\
    考试形式:%
    \setlength{\fboxsep}{-0.4pt}%
    \ifopen
      闭卷 \fbox{\rule[0.683em]{0.683em}{0em}}、%
      开卷 \rule{0.683em}{0.683em},%
    \else
      闭卷 \rule{0.683em}{0.683em}、%
      开卷 \fbox{\rule[0.683em]{0.683em}{0em}},%
    \fi
    \mbox{允许带\Fill{17em}{\Thing}入场}\\
    考生姓名:\Fill{5.2em}{}
    学号:\Fill{5.2em}{}
    专业:\Fill{5.2em}{}
    班级:\Fill{5.2em}{}
    \cjluexam@score{\totvalue{section}}%
    \endgroup
  }
\fi
%    \end{macrocode}
% \end{macro}
%
% \begin{macro}{\Fill}
% \begin{macro}{\cjluexam@length}
% 带下划线的指定长度文本.
% 文本长度小于指定长度时, 文本居中; 否则延长下划线.
% 参数 \opt{\#1} 是下划线的指定长度,
% 参数 \opt{\#2} 是指定的文本.
%
% \cmd{\cjluexam@length} 是辅助长度变量, 用于条件判断.
%
%    \begin{macrocode}
\newlength\cjluexam@length
\newcommand\Fill[2]{%
  \settowidth{\cjluexam@length}{#2}%
  \ifdim\cjluexam@length>#1\relax
    \underline{\mbox{#2}}%
  \else
    \underline{\makebox[#1]{#2}}%
  \fi
}
%    \end{macrocode}
% \end{macro}
% \end{macro}
%
% \begin{macro}{\cjluexam@score}
% 生成列数可变的得分栏.
%
% \changes{v1.0.2}{2021/11/16}{通过定义列格式简化操作}
%
%    \begin{macrocode}
\RequirePackage{tabularx}
\newcommand\cjluexam@score[1]{%
  \begin{flushleft}
    \edef\cjluexam@num{#1}%
    \ifnum\cjluexam@num<1\relax\def\cjluexam@num{1}\fi
    \ifnum\cjluexam@num<7\relax
      \newcolumntype{M}{|*{\numexpr\cjluexam@num+2}
      {>{\centering\arraybackslash}X|}}%
    \else
      \ifnum\cjluexam@num<11\relax
        \newcolumntype{M}{|c|*{\numexpr\cjluexam@num+1}
        {@{}>{\centering\arraybackslash}X@{}|}}%
      \else
        \newcolumntype{M}{|c|*{\cjluexam@num}
        {@{}>{\centering\arraybackslash}X@{}|}c|}%
      \fi
    \fi
    \begin{tabularx}{\textwidth}{M}
      \hline
      题序   & \cjluexam@Ncell{1}{\cjluexam@num}{2} & 总分 \\\hline
      得分   & \cjluexam@Ncell{1}{\cjluexam@num}{0} &      \\\hline
      评卷人 & \cjluexam@Ncell{1}{\cjluexam@num}{0} &      \\\hline
    \end{tabularx}%
  \end{flushleft}%
}
%    \end{macrocode}
% \end{macro}
%
% \begin{macro}{\cjluexam@Ncell}
% \changes{v1.0.2}{2021/11/16}{修正命名笔误}
% \begin{macro}{\cjluexam@gobble}
% 生成指定列数的单元格的格式字符串.
% 其中, 参数 \opt{\#1} 一定要取 \opt{1}, 表示起始序号.
% 参数 \opt{\#2} 是指定的列数.
% 参数 \opt{\#3} 取 \opt{0} 时表示单元格内部无内容;
% 取 \opt{1} 时输出递增数列;
% 取其他值时输出中文数字形式的递增数列.
%
% 辅助命令 \cmd{\cjluexam@gobble} 的功能就是丢弃它的参数.
%
% 该方案参考了
% \href{https://tex.stackexchange.com/questions/165625/how-to-fill-a-dynamically-generated-table-with-dynamic-content}{StackExchange}
% 与
% \href{https://www.overleaf.com/learn/latex/Articles/How_does_%5Cexpandafter_work:_A_detailed_macro_case_study}{Overleaf}
% 的代码.
% 其他方案可参考\href{https://zhuanlan.zhihu.com/p/58103339}{知乎}.
%
%    \begin{macrocode}
\newcommand\cjluexam@gobble[1]{}
\newcommand\cjluexam@Ncell[3]{%
  \ifnum  #1<2 \expandafter\cjluexam@gobble\fi &
  \ifcase #3 \or #1\else\zhnumber{#1}\fi
  \ifnum  #1<#2
    \expandafter\cjluexam@Ncell
    \expandafter{\the\numexpr1+#1\expandafter}%
    \expandafter{\the\numexpr  #2\expandafter}%
    \expandafter{\the\numexpr  #3\expandafter}%
  \fi
}
%    \end{macrocode}
% \end{macro}
% \end{macro}
%
% \begin{macro}{\cjluexam@seat}
% 打印座位号.
% 代码中的坐标以页面左上角为原点.
%
%    \begin{macrocode}
\RequirePackage{tikz}
\newcommand\cjluexam@seat{%
  \begin{tikzpicture}
    [remember picture,overlay,ampersand replacement=\&]
    \zihao{-4}
    \matrix[nodes={draw,inner ysep=8pt}]
      at ([shift={(25mm,-25mm)}]current page.north west)
      {\node{\phantom{座位号}}; \& \node{座位号};\\};
  \end{tikzpicture}%
}
%    \end{macrocode}
% \end{macro}
%
% \subsection{题目标题}
%
% 设置 \cmd{\section} 级标题的格式.
% 其中 \opt{runin=true} 表示标题后不换行, 直接排版后续内容.
%
%    \begin{macrocode}
\ctexset{section={
  name={,、},
  number=\chinese{section},
  format=\sffamily,
  aftername={},
  aftertitle={},
  beforeskip=1ex plus .5ex minus .5ex,
  afterskip =1ex plus .5ex minus .5ex,
  runin=true,
}}
%    \end{macrocode}
%
% \subsection{其他}
%
% 两行之间的最小空白.
%    \begin{macrocode}
\setlength\lineskiplimit{.25em}
\setlength\lineskip     {.25em}
%    \end{macrocode}
%
%    \begin{macrocode}
%</class>
%    \end{macrocode}
%
% \Finale
\endinput
