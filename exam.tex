% !TeX program = xelatex
\documentclass[oneside,AutoFakeBold=2.85]{cjluexam}
% 文档选项 oneside 与 twoside 分别用于单面打印、双面打印

% Information ----------------------------------------------------
\cjluexamset{
  %academicyear = {2020~2021},  % 学年
  %term         = {1},           % 学期
  %course       = {高等数学},    % 科目
  %type         = {A},           % 试卷类型
  %college      = {理学院},      % 开课学院
  %openbook,                     % 开卷
  %seat,                         % 座位号
  %carryon      = {计算器},      % 可带入考场的物品
}

% Custom ---------------------------------------------------------
%\setcounter{section}{9}         % 得分栏列数根据 section 计数器自动调整
%\ctexset{section/runin=true}    % 题目标题后不换行, 直接排版后续内容

% ----------------------------------------------------------------
\begin{document}

\maketitle

\section{单项选择题(每小题3分,共15分)}

\begin{enumerate}[topsep=8pt,itemsep=4pt]
  \item
    请选择\Fill{}.
    \begin{tasks}(4)
      \task 选项
      \task 选项
      \task 选项
      \task 选项
    \end{tasks}
  \item
    请选择\Fill{}.
    \begin{tasks}(2)
      \task 选项
      \task 选项
      \task 选项
      \task 选项
    \end{tasks}
  \item
    请选择\Fill{}.
    \begin{tasks}(1)
      \task 选项
      \task 选项
      \task 选项
      \task 选项
    \end{tasks}
  \item
    请选择\Fill{}.
    \begin{tasks}
      \task 选项
      \task 选项
      \task 选项
      \task 选项
    \end{tasks}
  \item
    请选择\Fill{}.
    \begin{tasks}(2)
      \task 选项
      \task 选项
      \task 选项
      \task 选项
    \end{tasks}
\end{enumerate}

\clearpage

\section{填空题(每小题3分,共15分)}

\begin{enumerate}[topsep=8pt, itemsep=4pt]
  \item
    请填空\Fill[8em]{}.
  \item
    请填空\Fill[8em]{}.
  \item
    请填空\Fill[8em]{}.
  \item
    请填空\Fill[8em]{}.
  \item
    请填空\Fill[8em]{}.
\end{enumerate}

\section{(每小题10分,共20分)}

\begin{enumerate}
  \item
    请计算
    \vspace{8cm}
  \item
    请计算
\end{enumerate}

\clearpage

\section{(每小题10分,共30分)}

\begin{enumerate}
  \item
    请解答
    \vfill
  \item
    请解答
    \vfill
  \item
    请解答
    \vfill
\end{enumerate}

\clearpage

\section{(每小题10分,共20分)}

\begin{enumerate}
  \item
    请解答
    \vfill
  \item
    请解答
    \vfill
\end{enumerate}

\end{document}
